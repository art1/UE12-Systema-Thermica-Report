\section{Introdution}


%%%%%%%%%%%%%%%%% TASK 1 %%%
\section{Thermal Dimensioning of SPOT Satellite Platform}

After having investigated the corresponding fluxes for the applied orbit in the two different cases, as well as the rejection capacities for each side of a cube satellite, we use these results to determine the thermical distribution inside the satellite platform of SPOT. For this, we extend the cube model used before by assuming that it contains 4 different equipments inside, as well as 4 radiators outside for each equipment. The remaining 2 sides of the cube will have no radiators, since we assume that the batteries and the payload are connected to those sides. Within Thermica, we start by using the geometric models for the cold case, SPOT-BOL, and end of life, SPOT-EOL. 

\subsection{Positioning of the equipment}
The very first important step of the analysis is to determine the best places to put the equipment, i.e. on which side of the cube depending on the equipment's specification and the fluxes calculated before. These specifications are listed in table \ref{tab:equipmentspecification}

\begin{table}[h!]
\centering
\begin{tabular}{ | c| c|c|c|c|c|c|c| }
\hline 
&  \multicolumn{2}{|c|}{Nominal Mode} &  \multicolumn{2}{|c|}{Safe Mode} & & \multicolumn{2}{|c|}{Dissipated Power [W] }\\
Function & Tmin & Tmax & Tmin & Tmax & Mass & nominal mode & safe mode  \\
& \multicolumn{2}{|c|}{$^{\circ}$C} & \multicolumn{2}{|c|}{$^{\circ}$C} & kg &Operation/Stand-By &  \\  \hline

Power Supply &-10   & 40  & -15&50  & 40  &  150 /50 & 20  \\ \hline
On-board Computer &-10  & 40 &-15 &50 &5 &15 /10& 10 \\ \hline
Attitude Control &-10  &40 & -15& 50&30 &110/50 &30  \\ \hline
Telemetry &-10  &40 &-15 &50 &20 &70/0 & 0  \\ \hline
\end{tabular}
\caption{Specifications for equipments}
\label{tab:equipmentspecification}
\end{table}

To place the equipment in the most efficient way, we exclude the $X$-axis, since the batteries, solar panels and the payload are going to be placed there. For the arrangement, we consider the rejection capacities of each of the sides in the hot case scenario and power consumptions of the equipments. Also, one has to consider the safe mode, which provides basic functions of the satellite by pointing the +$Z_{s}$ side towards the sun, which results in a strong solar flux. The resulting best arrangement is listed in table \ref{tab:arrangement}

\begin{table}[H]
\centering
\begin{tabular}{ | c| c|}
\hline 
Side & Equipment  \\ \hline
$-Y_{s}$ & Power Supply  \\ \hline
$-Z_{s}$ &On-board Computer \\\hline
$+Y_{s}$ &Attitude Control \\ \hline
$+Z_{s}$ &Telemetry   \\ \hline
\end{tabular}
\caption{Chosen arrangement of equipment}
\label{tab:arrangement}
\end{table}


\subsection{Sizing of Radiators}
Once  the positions of the equipment are defined, we are able to size the radiators according to the power consumptions and heat capacities. The required area can be calculated using $A=\frac{P_{diss}}{Q_{s,i}}$, where $Q_{s}$ are the rejection capacities for the individual sides at the hot case and $P_{diss}$ the power that is dissipated by the equipment.


\begin{table}[H]
\centering
\begin{tabular}{ | c| c| c| }
\hline 
Side & Equipment  & Areas [$m^{2}$] \\ \hline
$-Y_{s}$ & Power Supply & 0.49   \\ \hline
$-Z_{s}$ &On-board Computer & 0.06   \\\hline
$+Y_{s}$ &Attitude Control &  0.36 \\ \hline
$+Z_{s}$ &Telemetry  &    0.22 \\ \hline
\end{tabular}
\caption{Required radiator sizes in the hot case}
\label{tab:arrangement}
\end{table}

 
\subsection{Dimensioning of Heating Power}
After the sizing of the radiator areas is performed using the hot case, the additional heating power, that might be required in the cold case, has to be determined. For that, we calculate the power that is actually evacuated in the cold case using the heat capacities in the cold case and the radiator sizes calculated before. From that, we reduce the power that is dissipated by the equipment in the hot case as listed in table \ref{tab:equipmentspecification}. This difference is the power that is lost and results in a continuous cooling, so this difference in power has to be added by heating to avoid further cooling. We assume that the additional heating power avoids the equipment to drop below $-5^{\circ}$C, as defined in the cold case of the cube before Table lists the calculated evacuated power in this case, as well as the dissipated power in the hot case for each equipment and and the resulting required heating:

\begin{table}[H]
\centering
\begin{tabular}{ | c| c| c| c|c|}
\hline 
Direction& Equipment  & Evacuated Power [W] $ Dissipated Power [W] $ Heating Power [W] \\ \hline
$-Y_{s}$ & Power Supply &  50 & 79 & 29    \\ \hline
$-Z_{s}$ &On-board Computer &10 &6 &  0  \\\hline
$+Y_{s}$ &Attitude Control & 50 & 58 & 8  \\ \hline
$+Z_{s}$ &Telemetry  &  0   & 40 & 40 \\ \hline
\end{tabular}
\caption{Power calculations in cold case}
\label{tab:arrangement}
\end{table}

\section{Thermal Calculations for SPOT}
In order to test and verify our previous calculations, we simulate the thermal behaviour of our SPOT satellite using Thermica to proof that the satellite fulfills the requirements of staying within $-5^{\circ}$ and $40^{\circ}$. We used the geometric models in the hot case and in the cold case for the satellite SPOT4,  which were already provided. Since we have more information now about the satellites compared to the first simplified cube simulations, we also have to adapt several files throughout the simulation. 

\subection{Nodal network}
In thermica, interactions are described using the so called nodal network, where as an example the output of several simulation elements is linked to further simulation elements that need that data is input. Therefore, the nodal network gives an overview about the interactions of different simulation elemenets. In our case, the different equipments are assigned different node numbers, i.e. 3 to the $+Y_{s}$ side of the cube and thusd to the attitude control, 4 to the power supply side,  5 to the telemetry side and 6 to on-board computer. The $\pm X$ sides are left out since they are not considered in this thermical simulation as they are connected to the batteries and payload.

\subsection{Radiative Phenomenon}  



