\section{Introduction}
Thermal Control Systems of satellites are one of the major subsystems that contribute and sustain the life of a spacecraft. Thus, this lab is focusing on the application of various techniques in order to analyse first a simple geometric figure and then a more complex structure, the SPOT satellite (SSO LEO at 800km height). In general, the most crucial phases during a mission occur during the beginning and the end of the spacecraft lifetime, further denoted as EOL respectively BOL, and during the Hot and the Cold Case of the orbit.\\
To analyse these critical time frames, a Software developed by Airbus Defence \& Space, SYSTEMA, is used. SYSTEMA is a framework of different software packages, which include Thermica and Thermisol. These will be used in the further report to model, calculate and analyse the aforementioned geometric figures. \\In particular, the software breaks the analysis down into two main parts: the analysis of radiative phenomena during the orbits of the spacecraft and a nodal analysis based on a thermal equilibrium.

\section{Rejection Capacity Calculation}
\subsection{Geometric Model of a Cube in Thermica}
The first step to a thermal analysis is the modelling of our geometric figure, since at first we will analyse radiating surfaces. As stated before, in this part only a simplified geometric model of the SPOT satellite is used to familiarise with and gain a basic understanding of Thermica/Thermisol.\\
These simplified surfaces are supposed to be coated with aluminised SSM with an emissivity of 0.78 and an absorptivity of 0.15 at BOL and 0.19 at EOL (5 years).
 
\subsection{Solar, planetary IR and Albedo Flux}
After having obtained a geometric model, the mission parameters have to be defined. In particular, these parameter contain the trajectory as well as the kinematics of the previously defined geometric shape.\\
To specify these parameters, Thermica offers the option to create a \textit{mission}, i.e. an orbit with specified parameters: orbit height of 830km, Sun-synchronous with the ascending node at 10:30 pm local time




\subsection{Rejection Capacity}


\section{Thermal Dimensioning of SPOT Satellite Platform}

After having investigated the corresponding fluxes for the applied orbit in the two different cases, as well as the rejection capacities for each side of a cube satellite, we use these results to determine the thermical distribution inside the satellite platform of SPOT. For this, we extend the cube model used before by assuming that it contains 4 different equipments inside, as well as 4 radiators outside for each equipment. The remaining 2 sides of the cube will have no radiators, since we assume that the batteries and the payload are connected to those sides. Within Thermica, we start by using the geometric models for the cold case, SPOT-BOL, and end of life, SPOT-EOL. 

\subsection{Positioning of the equipment}
The very first important step of the analysis is to determine the best places to put the equipment, i.e. on which side of the cube depending on the equipment's specification and the fluxes calculated before. These specifications are listed in table \ref{tab:equipmentspecification}

\begin{table}[h!]
\centering
\begin{tabular}{ | c| c|c|c|c|c|c|c| }
\hline 
&  \multicolumn{2}{|c|}{Nominal Mode} &  \multicolumn{2}{|c|}{Safe Mode} & & \multicolumn{2}{|c|}{Dissipated Power [W] }\\
Function & Tmin & Tmax & Tmin & Tmax & Mass & nominal mode & safe mode  \\
& \multicolumn{2}{|c|}{$^{\circ}$C} & \multicolumn{2}{|c|}{$^{\circ}$C} & kg &Operation/Stand-By &  \\  \hline

Power Supply &-10   & 40  & -15&50  & 40  &  150 /50 & 20  \\ \hline
On-board Computer &-10  & 40 &-15 &50 &5 &15 /10& 10 \\ \hline
Attitude Control &-10  &40 & -15& 50&30 &110/50 &30  \\ \hline
Telemetry &-10  &40 &-15 &50 &20 &70/0 & 0  \\ \hline
\end{tabular}
\caption{Specifications for equipments}
\label{tab:equipmentspecification}
\end{table}

To place the equipment in the most efficient way, we exclude the $X$-axis, since the batteries, solar panels and the payload are going to be placed there. For the arrangement, we consider the rejection capacities of each of the sides in the hot case scenario and power consumptions of the equipments. Also, one has to consider the safe mode, which provides basic functions of the satellite by pointing the +$Z_{s}$ side towards the sun, which results in a strong solar flux. The resulting best arrangement is listed in table \ref{tab:arrangement}

\begin{table}[H]
\centering
\begin{tabular}{ | c| c|}
\hline 
Side & Equipment  \\ \hline
$-Y_{s}$ & Power Supply  \\ \hline
$-Z_{s}$ &On-board Computer \\\hline
$+Y_{s}$ &Attitude Control \\ \hline
$+Z_{s}$ &Telemetry   \\ \hline
\end{tabular}
\caption{Chosen arrangement of equipment}
\label{tab:arrangement}
\end{table}


\subsection{Sizing of Radiators}
Once  the positions of the equipment are defined, we are able to size the radiators according to the power consumptions and heat capacities. The required area can be calculated using $A=\frac{P_{diss}}{Q_{s,i}}$, where $Q_{s}$ are the rejection capacities for the individual sides at the hot case and $P_{diss}$ the power that is dissipated by the equipment.


\begin{table}[H]
\centering
\begin{tabular}{ | c| c| c| }
\hline 
Side & Equipment  & Areas [$m^{2}$] \\ \hline
$-Y_{s}$ & Power Supply & 0.49   \\ \hline
$-Z_{s}$ &On-board Computer & 0.06   \\\hline
$+Y_{s}$ &Attitude Control &  0.36 \\ \hline
$+Z_{s}$ &Telemetry  &    0.22 \\ \hline
\end{tabular}
\caption{Required radiator sizes in the hot case}
\label{tab:arrangement}
\end{table}

 
\subsection{Dimensioning of Heating Power}
After the sizing of the radiator areas is performed using the hot case, the additional heating power, that might be required in the cold case, has to be determined. For that, we calculate the power that is actually evacuated in the cold case using the heat capacities in the cold case and the radiator sizes calculated before. From that, we reduce the power that is dissipated by the equipment in the hot case as listed in table \ref{tab:equipmentspecification}. This difference is the power that is lost and results in a continuous cooling, so this difference in power has to be added by heating to avoid further cooling. We assume that the additional heating power avoids the equipment to drop below $-5^{\circ}$C, as defined in the cold case of the cube before Table lists the calculated evacuated power in this case, as well as the dissipated power in the hot case for each equipment and and the resulting required heating:

\begin{table}[H]
\centering
\begin{tabular}{ | c| c| c| c|c|}
\hline 
Direction& Equipment  & Evacuated Power [W] $ Dissipated Power [W] $ Heating Power [W] \\ \hline
$-Y_{s}$ & Power Supply &  50 & 79 & 29    \\ \hline
$-Z_{s}$ &On-board Computer &10 &6 &  0  \\\hline
$+Y_{s}$ &Attitude Control & 50 & 58 & 8  \\ \hline
$+Z_{s}$ &Telemetry  &  0   & 40 & 40 \\ \hline
\end{tabular}
\caption{Power calculations in cold case}
\label{tab:arrangement}
\end{table}

\section{Thermal Calculations for SPOT}
In order to test and verify our previous calculations, we simulate the thermal behaviour of our SPOT satellite using Thermica to proof that the satellite fulfills the requirements of staying within $-5^{\circ}$ and $40^{\circ}$. We used the geometric models in the hot case and in the cold case for the satellite SPOT4,  which were already provided. Since we have more information now about the satellites compared to the first simplified cube simulations, we also have to adapt several files throughout the simulation. 

\subsection{Nodal network}
In thermica, interactions are described using the so called nodal network, where as an example the output of several simulation elements is linked to further simulation elements that need that data is input. Therefore, the nodal network gives an overview about the interactions of different simulation elemenets. In our case, the different equipments are assigned different node numbers, i.e. 3 to the $+Y_{s}$ side of the cube and thusd to the attitude control, 4 to the power supply side,  5 to the telemetry side and 6 to on-board computer. The $\pm X$ sides are left out since they are not considered in this thermical simulation as they are connected to the batteries and payload.

\subsection{Radiative Phenomenon}
In the thermal simulation we consider the hot and the cold case, end of life and beginning of life. For setting up the simulation, several elements defining the mission, which depends on the case, the nodal descriptin, "Radiation", "Solar fluxes", "Planetary fluxes, "Skeleton" and finally "Solver" have to be connected within Thermica. However, we need to adapt several files before using the latter two elements in the simulation, so these are deactivated for the first run of the simulation.  
The reason for that extra step is that the created file called "Spot.usr.nwk", which contains information about the radiative interactions of the several elements, assumes e.g. a radiator surface area of $1$ m$^{2}$ and other different values. However, to adapt the simulation to our case, we have to change these values in the file. In order to do so, one has to add the various values calculated above like the different fluxes for each radiator, heat capacities and dissipations. Once these changes are done, one can run the simulation with the last steps including the Thermica solver THERMOSOL.

